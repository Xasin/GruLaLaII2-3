
Der Versuch \textit{Leistung im Wechselstrom} erläuterte die verschiedenen Arten der Leistung, welche im Wechselstrom im Vergleich zum Gleichstrom auftreten können. Diese wurden im ersten Versuchsteil anhand einiger Messungen der Momentanleistungen mit einem trägheitsfreien Oszilloskop dargestellt. Die Leistung an verschiedenen Verbrauchern wurde untersucht und die Addierbarkeit der einzelnen Leistungen wurde grafisch und mathematisch bewiesen. Es wurde die Blindleistungskompensation anhand eines LCR-Schwingkreises in Resonanz erläutert. Zusätzlich wurde die Leistung einer nichtlinearen Belastung betrachtet und eine simple Fourier-Analyse der Leistung durchgeführt.

Im zweiten Versuchsteil wurden die Wirk-, Blind- und Scheinleistungen verschiedener Belastungen mit trägen, mittelwertsbildenden Messgeräten aufgenommen. Die Funktionsweise der jeweiligen Messchaltung wurde erläutert und die gemessenen und berechneten Werte wurden miteinander verglichen. Unstimmigkeiten der gemessenen Werte wurden auf passende Fehlerquellen zurückgeführt. Schlussendlich wurde die Scheinleistung der jeweiligen Belastung auf verschiedene Arten berechnet und die Werte wurden miteinander verglichen.